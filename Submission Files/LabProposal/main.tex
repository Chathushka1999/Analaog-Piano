\documentclass[11pt]{article}

\usepackage{geometry}
 \geometry{
 a4paper,
 total={170mm,257mm},
 left=25mm,
 top=25mm,
 }

\title{Piano}
\author{Proposal : Group 22 }
\date{}
\date{\vspace{-5ex}}

\begin{document}
\maketitle
\section*{Description}
The piano can be considered as a simple starting point of fusing music into electronics.
The purpose of this project is to develop a piano with seven notes with only analog
electronic circuits.

\section*{Outcomes}
Design a piano that can generate sounds of frequencies associated with all the seven notes which
are sensible for the human ear. The piano sound must be amplified for the sake of making
it audible clearly. It is expected that the sound to be smoothened with noise-free
constraints to make it more enjoyable.
\subsection*{Skills}
\begin{enumerate}
    \item Hands-on experience with analog wave generation and wave addition.
    \item Designing amplifier circuits from scratch which can handle higher frequencies.
\end{enumerate}
\subsection*{Evaluation Criteria}
\begin{enumerate}
    \item Design: Neatness, and sustainability of the electronic circuit.
    \item Smoothness and audible level of output.
    \item The smoothness of sound during switching between notes.
\end{enumerate}

\section*{Additional Notes}
\begin{enumerate}
    \item Members are not allowed to use any kind of IC, Timers, or op-amps in the design of the project.
    \item Members are expected to have proper noise-cancelling strategies to make the sound enjoyable.
    \item Having a proper sound controlling mechanism to switch between sounds is considered a plus.
    \item Members are expected to implement techniques of pure wave generation and wave addition to achieve these requirements.
\end{enumerate}
\end{document}
